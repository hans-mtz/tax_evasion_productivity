% Options for packages loaded elsewhere
\PassOptionsToPackage{unicode}{hyperref}
\PassOptionsToPackage{hyphens}{url}
\PassOptionsToPackage{dvipsnames,svgnames,x11names}{xcolor}
%
\documentclass[
  12pt]{article}

\usepackage{amsmath,amssymb}
\usepackage{iftex}
\ifPDFTeX
  \usepackage[T1]{fontenc}
  \usepackage[utf8]{inputenc}
  \usepackage{textcomp} % provide euro and other symbols
\else % if luatex or xetex
  \usepackage{unicode-math}
  \defaultfontfeatures{Scale=MatchLowercase}
  \defaultfontfeatures[\rmfamily]{Ligatures=TeX,Scale=1}
\fi
\usepackage{lmodern}
\ifPDFTeX\else  
    % xetex/luatex font selection
\fi
% Use upquote if available, for straight quotes in verbatim environments
\IfFileExists{upquote.sty}{\usepackage{upquote}}{}
\IfFileExists{microtype.sty}{% use microtype if available
  \usepackage[]{microtype}
  \UseMicrotypeSet[protrusion]{basicmath} % disable protrusion for tt fonts
}{}
\makeatletter
\@ifundefined{KOMAClassName}{% if non-KOMA class
  \IfFileExists{parskip.sty}{%
    \usepackage{parskip}
  }{% else
    \setlength{\parindent}{0pt}
    \setlength{\parskip}{6pt plus 2pt minus 1pt}}
}{% if KOMA class
  \KOMAoptions{parskip=half}}
\makeatother
\usepackage{xcolor}
\setlength{\emergencystretch}{3em} % prevent overfull lines
\setcounter{secnumdepth}{5}
% Make \paragraph and \subparagraph free-standing
\ifx\paragraph\undefined\else
  \let\oldparagraph\paragraph
  \renewcommand{\paragraph}[1]{\oldparagraph{#1}\mbox{}}
\fi
\ifx\subparagraph\undefined\else
  \let\oldsubparagraph\subparagraph
  \renewcommand{\subparagraph}[1]{\oldsubparagraph{#1}\mbox{}}
\fi


\providecommand{\tightlist}{%
  \setlength{\itemsep}{0pt}\setlength{\parskip}{0pt}}\usepackage{longtable,booktabs,array}
\usepackage{calc} % for calculating minipage widths
% Correct order of tables after \paragraph or \subparagraph
\usepackage{etoolbox}
\makeatletter
\patchcmd\longtable{\par}{\if@noskipsec\mbox{}\fi\par}{}{}
\makeatother
% Allow footnotes in longtable head/foot
\IfFileExists{footnotehyper.sty}{\usepackage{footnotehyper}}{\usepackage{footnote}}
\makesavenoteenv{longtable}
\usepackage{graphicx}
\makeatletter
\def\maxwidth{\ifdim\Gin@nat@width>\linewidth\linewidth\else\Gin@nat@width\fi}
\def\maxheight{\ifdim\Gin@nat@height>\textheight\textheight\else\Gin@nat@height\fi}
\makeatother
% Scale images if necessary, so that they will not overflow the page
% margins by default, and it is still possible to overwrite the defaults
% using explicit options in \includegraphics[width, height, ...]{}
\setkeys{Gin}{width=\maxwidth,height=\maxheight,keepaspectratio}
% Set default figure placement to htbp
\makeatletter
\def\fps@figure{htbp}
\makeatother

\addtolength{\oddsidemargin}{-.5in}%
\addtolength{\evensidemargin}{-1in}%
\addtolength{\textwidth}{1in}%
\addtolength{\textheight}{1.7in}%
\addtolength{\topmargin}{-1in}%

\usepackage{amsthm}
\newtheorem{ass}{Assumption}
% \usepackage{mathtools}
\usepackage[final, nomargin, inline, nomarginclue, author=HM]{fixme}
% \fxusetargetlayout{color}
\fxsetface{inline}{\color{blue}}
\fxsetface{env}{\color{blue}}
% \usepackage{amsmath}
% \usepackage[final,nomargin,index,inline, author=]{fixme}
% \usepackage{bbm}
\usepackage{unicode-math}
\usepackage{tabularx}
\usepackage{adjustbox}
% \usepackage{amsmath}
% - \usepackage{longtable}
% - \usepackage{booktabs}
% - \usepackage{graphicx}
% \newtheorem{definition}{Definition}
% \newtheorem{theo}{Theorem}
% \newtheorem{lemma}{Lemma}
% \newtheorem{ass}{Assumption}
% \usepackage{xkvltxp}
% -  #final or draft
% - \fxsetup{envlayout=color, targetlayout=color}
% - \fxsetface{inline}{\color{blue}}

% kableExtra required packages:
\usepackage{booktabs}
\usepackage{longtable}
% \usepackage{array}
% \usepackage{multirow}
% \usepackage{wrapfig}
% \usepackage{float}
% \usepackage{colortbl}
% \usepackage{pdflscape}
% \usepackage{tabu}
\usepackage{threeparttable}
\usepackage{threeparttablex}
% \usepackage[normalem]{ulem}
% \usepackage{makecell}
% \usepackage{xcolor}
\usepackage{booktabs}
\usepackage{longtable}
\usepackage{array}
\usepackage{multirow}
\usepackage{wrapfig}
\usepackage{float}
\usepackage{colortbl}
\usepackage{pdflscape}
\usepackage{tabu}
\usepackage{threeparttable}
\usepackage{threeparttablex}
\usepackage[normalem]{ulem}
\usepackage{makecell}
\usepackage{xcolor}
\makeatletter
\@ifpackageloaded{caption}{}{\usepackage{caption}}
\AtBeginDocument{%
\ifdefined\contentsname
  \renewcommand*\contentsname{Table of contents}
\else
  \newcommand\contentsname{Table of contents}
\fi
\ifdefined\listfigurename
  \renewcommand*\listfigurename{List of Figures}
\else
  \newcommand\listfigurename{List of Figures}
\fi
\ifdefined\listtablename
  \renewcommand*\listtablename{List of Tables}
\else
  \newcommand\listtablename{List of Tables}
\fi
\ifdefined\figurename
  \renewcommand*\figurename{Figure}
\else
  \newcommand\figurename{Figure}
\fi
\ifdefined\tablename
  \renewcommand*\tablename{Table}
\else
  \newcommand\tablename{Table}
\fi
}
\@ifpackageloaded{float}{}{\usepackage{float}}
\floatstyle{ruled}
\@ifundefined{c@chapter}{\newfloat{codelisting}{h}{lop}}{\newfloat{codelisting}{h}{lop}[chapter]}
\floatname{codelisting}{Listing}
\newcommand*\listoflistings{\listof{codelisting}{List of Listings}}
\makeatother
\makeatletter
\makeatother
\makeatletter
\@ifpackageloaded{caption}{}{\usepackage{caption}}
\@ifpackageloaded{subcaption}{}{\usepackage{subcaption}}
\makeatother
\makeatletter
\@ifpackageloaded{tcolorbox}{}{\usepackage[many]{tcolorbox}}
\makeatother
%%%% ---foldboxy preamble ----- %%%%%

\definecolor{fbx-default-color1}{HTML}{c7c7d0}
\definecolor{fbx-default-color2}{HTML}{a3a3aa}

\definecolor{fbox-color1}{HTML}{c7c7d0}
\definecolor{fbox-color2}{HTML}{a3a3aa}

% arguments: #1 typelabelnummer: #2 titel: #3
\newenvironment{fbx}[3]{\begin{tcolorbox}[enhanced, breakable,%
attach boxed title to top*={xshift=1.4pt},
boxed title style={boxrule=0.0mm, fuzzy shadow={1pt}{-1pt}{0mm}{0.1mm}{gray}, arc=.3em, rounded corners=east, sharp corners=west}, colframe=#1-color2, colbacktitle=#1-color1, colback = white, coltitle=black,  titlerule=0mm, toprule=0pt, bottomrule=.7pt, leftrule=.3em, rightrule=0pt, outer arc=.3em,  arc=0pt,	 sharp corners = east, left=.5em, bottomtitle=1mm, toptitle=1mm,title=\textbf{#2}\hspace{0.5em}{#3}]}
{\end{tcolorbox}}

% boxed environment with right border
\newenvironment{fbxSimple}[3]{\begin{tcolorbox}[enhanced, breakable,%
attach boxed title to top*={xshift=1.4pt},
boxed title style={boxrule=0.0mm, fuzzy shadow={1pt}{-1pt}{0mm}{0.1mm}{gray}, arc=.3em, rounded corners=east, sharp corners=west}, colframe=#1-color2, colbacktitle=#1-color1, colback = white, coltitle=black,  titlerule=0mm, toprule=0pt, bottomrule=.7pt, leftrule=.3em, rightrule=.7pt, outer arc=.3em,  	left=.5em, right=.5em, bottomtitle=1mm, toptitle=1mm,title=\textbf{#2}\hspace{0.5em}{#3}]}
{\end{tcolorbox}}

%%%% --- end foldboxy preamble ----- %%%%%
%%==== colors from yaml ===%
\definecolor{Assumption-color1}{HTML}{99CCFF}
\definecolor{Assumption-color2}{HTML}{FFFFFF}
\definecolor{Proposition-color1}{HTML}{99CCFF}
\definecolor{Proposition-color2}{HTML}{FFFFFF}
\definecolor{Definition-color1}{HTML}{99CCFF}
\definecolor{Definition-color2}{HTML}{FFFFFF}
\definecolor{Theorem-color1}{HTML}{99CCFF}
\definecolor{Theorem-color2}{HTML}{FFFFFF}
%=============%
\ifLuaTeX
  \usepackage{selnolig}  % disable illegal ligatures
\fi
\usepackage[]{natbib}
\bibliographystyle{agsm}
\usepackage{bookmark}

\IfFileExists{xurl.sty}{\usepackage{xurl}}{} % add URL line breaks if available
\urlstyle{same} % disable monospaced font for URLs
\hypersetup{
  pdftitle={Tax Evasion and Productivity},
  pdfauthor={Hans Martinez},
  pdfkeywords={Tax Evasion, Cost Overreporting, Production Function
Estimation, Productivity},
  colorlinks=true,
  linkcolor={blue},
  filecolor={Maroon},
  citecolor={Blue},
  urlcolor={Blue},
  pdfcreator={LaTeX via pandoc}}


\begin{document}


\def\spacingset#1{\renewcommand{\baselinestretch}%
{#1}\small\normalsize} \spacingset{1}


%%%%%%%%%%%%%%%%%%%%%%%%%%%%%%%%%%%%%%%%%%%%%%%%%%%%%%%%%%%%%%%%%%%%%%%%%%%%%%

\date{September 3, 2025}
\title{\bf Tax Evasion and Productivity}
\author{
Hans Martinez\thanks{email: hmarti33@uwo.ca.}\\
Department of Economics, University of Western Ontario\\
}
\maketitle

\bigskip
\bigskip
\begin{abstract}
I propose a novel method to estimate corporate tax evasion through cost
overreporting using production functions. I first show how that cost
overreporting leads to biased estimates of productivity and the
production function parameters. Then, I demonstrate how to recover
unbiased estimates of both in the presence of tax evasion. With these
estimates on hand, the production function can be inverted to recover
the true inputs. Cost overreporting can be estimated as the difference
between the observed and the true inputs. I apply this method to a
well-known dataset where I find evidence suggesting that cost
overreporting is widespread and quantitatively large. My results
indicate that ignoring cost overreporting leads to consistently larger
elasticities of intermediate inputs. The bias on the intermediate inputs
spreads to the elasticities of labor and capital, whose bias direction
varies by industry. Finally, I find significant differences in the
productivity distributions.
\end{abstract}

\noindent%
{\it Keywords:} Tax Evasion, Cost Overreporting, Production Function
Estimation, Productivity
\vfill

\newpage
\spacingset{1.9} % DON'T change the spacing!

\section*{Introduction}\label{introduction}
\addcontentsline{toc}{section}{Introduction}

I propose a novel method to estimate corporate tax evasion through cost
overreporting using production functions. I first show how that cost
overreporting lead to biased estimates of productivity and production
function parameters. Then, I show how to recover unbiased estimates of
the production function and productivity in the presence of tax evasion.
With the production function parameters and productivity on hand, the
production function can be inverted to recover the true inputs. Then, I
estimate cost overreporting as the difference between observed and true
inputs. I apply this method to a well-known dataset where I find
evidence suggesting that cost overreporting is widespread and
quantitatively large. I also find that ignoring cost overreporting leads
to consistently larger elasticities of intermediate inputs. The bias on
the intermediate inputs spreads to the elasticities of labor and
capital, whose bias direction varies by industry.

Corporate tax evasion through cost overreporting spreads globally
causing governments significant revenue losses. Cost overreporting,
however, has been largely overlooked by the literature except for a few
recent studies. The literature has mostly relied on manually detecting
evaders using administrative data or on experiments where researchers
randomly send letters and observing how firms adjust their tax
declarations. My approach is complementary to these methods. It is not
restricted to administrative data and can be applied using more commonly
available data such as firm-level surveys.

Cost overreporting arises when firms acquire false invoices to claim
additional tax deductions on value-added (VAT) and corporate income
taxes (CIT). According to the OECD's document \citet{OECD2017}, cost
overreporting --- also known as ``fake invoicing'', ``ghost firms'',
``invoice mills'', or ``missing traders''--- permeates internationally.
Reports from Latin America, Eastern Europe, Asia, and Africa claim cost
overreporting led to annual tax revenue losses as large as 5.6\% of the
GDP, for example, in Poland, 2016 (Poland's Minister of Finance,
2018)\footnote{Other reports show that cost overreporting led to revenue
  losses of 0.2\% of Chile's GDP in 2004 (Gonzalez and Velasquez, 2013;
  Jorrat, 2001; CIAT, 2008); 0.2\% of Colombia's GDP (Portafolio, 2019);
  and 0.03\% of Mexico's GDP in 2018 (Senado de la Republica, 2019).}

Despite its relevance, cost overreporting has been mostly overlooked by
the literature. On one hand, the few studies on this evasion strategy
exploit detailed administrative data \citep{Zumaya2021, Carrillo2022}.
Government tax authorities restrict access to administrative data
because of firms' confidentiality concerns. On the other hand, to the
best of my knowledge, no study has attempted to structurally identify
cost overreporting. Unlike the case of individuals
\citep{Pissarides1989, Paulus2015}, when it comes to corporate tax
evasion, researchers have to account for an additional source of
unobserved heterogeneity, productivity. Why? Because cost overreporting
might be naively quantified as low productivity. Intuitively, for a
given output level, high input utilization by a firm could be explained
by either the amount of input the firm overreports to evade taxes or by
a negative productivity shock.

To address this gap in the literature, first I formally show that
ignoring tax evasion leads to downward biased productivity estimates. I
then provide a new estimation strategy using production functions to
jointly recover the densities of tax evasion and productivity. The
intuition works as follows. In the absence of tax evasion, the
first-order conditions of the firms' cost-minimization problem let us
recover the common technology, the production function. Consequently, in
the presence of cost overreporting, deviations from this common
technology identify tax evasion up to the measurement error. Then, from
a subset of non-overreporting firms, the strategy identifies the
production function parameters and the density of the output shock.
Finally, using non-parametric deconvolution techniques, I jointly
recover the distributions of tax evasion and productivity.

Applying the method using firm-level data from Colombia between 1981 and
1991 ---a commonly used dataset in the production function
literature---, I find evidence suggesting that firms in four of the top
five industries (8 of the top 20) engage in cost overreporting. These
firms overreport up to 25\% of their costs. My estimates suggest that
the tax evading firms in the top 20 revenue industries caused the
government of Colombia approximately XXXX in tax revenue losses.
{[}exporters/importers/proprietorships/limited liability companies{]}
are {[}less/more{]} likely to engage in cost overreporting. I also find
that ignoring cost overreporting leads to consistently larger
elasticities of intermediate inputs by {[}what factor{]}. The bias on
the intermediate inputs spreads to the elasticities of labor and
capital, whose bias direction varies by industry {[}what range?{]}.
Lastly, I find significant differences in the productivity
distributions. In particular, true productivity distributions are
{[}how{]}. The differences between {[}exporters/corporations{]} and
{[}importers/limited liability companies{]} are {[}what?{]}.

\begin{anfxnote}{Remove}
Recently, evidence from Ecuador \citep{Carrillo2022} shows that cost
overreporting disseminates across all types of firms; constitutes a
quantitatively large share of firms' total deductions; and occurs mostly
in the intermediate inputs. Contrary to the literature consensus,
evasion by overreporting is not limited to small, semi-formal firms, but
large and formal firms overreport too. Likewise, firms were found to
overreport up to 14.1\% of the value of their purchase deductions.
Finally, when challenged by the tax authority, firms adjusted mostly
intermediate inputs.

The evidence from Ecuador also shows that big firms do not overreport
inputs. This is unsurprising for several reasons. First, a large firm
arguably draws more attention from the tax authority. Given its limited
resources, the government optimizes its expected revenues by targeting
the firms with the higher potential tax recovery, the few big ones.
Second, the cost of being caught cheating is potentially higher for big
firms. Large firms are likely to participate in international markets. A
tax evasion scandal in Colombia, for example, might affect US sales.
Finally, big firms potentially have more sophisticated strategies for
tax evasion \citep[e.g., profit shifting][]{Bustos2022}.

In a preliminary application, I show how to use the method to learn
about tax evasion even when we do not know the subset of non-evading
firms, but there has been a change in fiscal policy that incentivizes
cost overreporting. Using firm-level data from Colombia between 1981 and
1991, I show that tax evasion through cost overreporting increased after
the fiscal reform of 1983 between 8 and 9\% in 1985 and 1986. This
result stands at odds with previous studies that indicate that the
evasion of income tax and VAT declined during this period (Sanchez \&
Gutierrez, 1994)

I formally show that ignoring tax evasion leads to biased estimates of
productivity, which can be significant considering cost overreporting is
widespread and quantitatively large. In the proxy variable literature,
productivity is measured as the residual of a production function, where
the output is a function of the inputs. A key assumption is that input
demand is strictly monotonic on the productivity
\citep{Gandhi2020, Ackerberg2015, Levinsohn2003}. In other words, we
expect highly productive firms will use fewer inputs and produce more
output. When firms overreport their costs (inputs) to reduce their tax
liabilities, their reported inputs are higher than their actual
utilization, resulting in lower productivity estimates.

Tax evasion through cost overreporting has been largely overlooked by
the literature, though; most recent studies focus on revenue
underreporting.

Despite its relevance, detecting and measuring tax evasion ---through
cost overreporting or otherwise--- remains a non-trivial task even for
governments with detailed administrative data; mainly because of firms'
---and individuals'--- incentives to avoid getting caught. Direct
empirical measures are mostly unreliable because firms and individuals
have incentives to conceal their behavior \citep{Slemrod2019}. Hence, it
is unlikely that evasion would be truthfully reported in surveys, for
instance. Indirect structural measures have had some degree of success
in the case of individual income tax evasion \citep{Pissarides1989}, but
in the case of corporate tax evasion researchers must account for an
additional latent variable, the productivity of firms. The reason is
that tax evasion might be naively quantified as low productivity.
Intuitively, for a given level of output, high input utilization by a
firm could be explained by either the amount of input the firm
overreports to evade taxes or by a low productivity shock. In certain
countries, governments take advantage of the different sources of
administrative data. For example, in Ecuador, the tax authority uses
third-party information on reported corporate taxes to detect evasion
through revenue underreporting \citep{Carrillo2017}. However, access to
this type of administrative data is generally restricted for most
researchers.

Despite the vast efforts of the literature, measuring tax evasion
remains a non-trivial task\footnote{The challenge is to obtain reliable
  data. Practitioners usually have to undergo time- and
  resource-consuming processes to access or collect data; for example,
  by requesting access to government administrative data or by
  collecting their own through surveys or experiments.}.

Tax evasion is a significant concern for developing and developed
countries\footnote{Tax evasion has been a long-standing concern for
  developing countries, but since the 2008 economic crisis, it has also
  been of increasing importance for developed countries
  \citep{Slemrod2019}.}.

Having productivity estimates could potentially help, however, tax
evasion has not been satisfactorily addressed in the estimation of
productivity. I show that ignoring tax evasion leads to biased estimates
of productivity. To address this gap in the literature, I provide a new
estimation strategy to recover tax evasion and productivity using
commonly available data.

Indirect structural attempts to estimate corporate tax evasion have to
account for the productivity of the firms. The reason is that low
productivity might be naively quantified as tax evasion. Consider the
case in which firms overreport input expenses to reduce their profits to
evade taxes. A practitioner with output and input data might conclude
that output is a function of inputs, thus it can be considered a second
measure. She might try to recover true inputs and get an estimate of tax
evasion by the difference between reported and true inputs. However, for
a given level of output, high input utilization by a firm could be
explained by either the amount of input the firm overreports to evade
taxes or by its low productivity.

Firm-level estimates of productivity could be helpful to measure tax
evasion, however, tax evasion has not been satisfactorily addressed in
the estimation of productivity by the literature. The literature has
coped with tax evasion by treating it as a classical measurement error
\citep[e.g.,][p.204]{Blalock2004}. In other words, it has been assumed
that tax-evasion misreporting has zero mean ---some firms under-report
and others over-report so that the misreporting does not bias the
estimates of interest--- and this misreporting is independent of
everything else, including the attributes of the firms. However, the
classical measurement error argument is inconsistent with economic
intuition\footnote{Economic intuition will inform us that systematic
  misreporting due to tax evasion should lead firms to decrease profits
  by either underreporting sales or overreporting costs. Put
  differently, there's no economic incentive for firms to go the other
  direction ---overreporting sales or underreporting input expenses---
  and artificially increase their profits. An artificial increment of
  profits will increase the tax liabilities of the firm and decrease
  their after-tax real profits. Therefore, it is unlikely that tax
  evasion is mean zero. Economic intuition will also point out that the
  degree of misreporting is likely to vary across firms depending on the
  characteristics of the firms, e.g., age, size, owner's risk aversion,
  etc.}.

The contribution to the tax evasion literature is two-fold: (1) I
provide an estimation strategy using commonly available data, and (2)
the method identifies tax evasion through cost overreporting, an
overlooked by the literature but relevant phenomenon. The contribution
to the productivity literature is to show that ignoring tax evasion
leads to biased estimates and to provide a method to recover the
productivity density in the presence of input overreporting.

Why do we care about productivity measurement bias due to tax evasion?
First, it might help explain part of the productivity gap puzzle.
Economists have found \emph{enormous and consistent} productivity
differences across producers \citep{Syverson2011}. We still don't know
how much of this productivity gap ---the difference between the lower
and higher percentiles of the distribution--- tax evasion can explain.
Second, it should be considered in the design of public policies aiming
at reducing resource misallocations. Aggregate productivity ---and,
thus, the economic growth of a country and the welfare of its
citizens--- depends on the efficient allocation of its inputs among its
firms. Ideally, resources should be allocated to the most productive
firms. Additionally, it is not necessarily the case that firms with the
highest incentives to evade taxes are always the lowest productive. The
incentives depend on the tax system. For example, in countries where a
profit threshold determines different tax rates, the firms near the
threshold are the ones with the highest incentives to misreport, but the
threshold might be completely unrelated to productivity. Therefore,
firm-level productivity estimates adjusted by tax evasion misreporting
might inform better the design of public policies aiming at an efficient
reallocation of resources to boost economic growth.

Moreover, accounting for tax evasion in estimating and measuring
firm-level productivity is particularly significant for developing
countries. This is the case because tax evasion is likely to be higher
in low- and middle-income countries\footnote{Informality ---the extreme
  form of tax evasion--- is larger in developing countries
  \citep{Loayza2006, LaPorta2014}.}, and because the productivity gap in
these countries is wider ---wider productivity gaps are associated with
higher resource misallocations. For example, recent estimates of
non-detected tax evasion are up to \$10 billion USD per year in Mexico
\citep{Zumaya2021}, approximately 1\% of the country's GDP. Furthermore,
some studies argue that input misallocation ---implied by a wider
productivity gap--- explains a significant part of the differences in
income per capita between developed and developing countries
\citep{Syverson2011, Levy2018}. For instance, evidence from Colombia
suggests that labor and financial policy reforms during the 1990s aimed
at reallocating away from low- and towards high-productivity firms
---effectively reducing the productivity gap--- resulted in a higher
aggregate productivity \citep{Eslava2004}.

The main challenge of dealing with tax evasion is that it is hidden by
nature, for that reason,

So, then, in this paper, I ask,

\begin{quote}
can we recover unbiased productivity estimates at the firm level using a
gross-output production function in the presence of systematic
misreporting due to tax evasion? if so, what is the magnitude of this
bias, in particular for developing countries? given that different tax
systems ---rates, rules and enforcement procedures--- generate different
tax evasion incentives, how does the producitivity bias vary according
to different tax systems? how much of the productivity gap can tax
evasion explain within a country and across countries?
\end{quote}

To answer these questions, this paper studies productivity at the
establishment level for Mexican and Colombian firms accounting for tax
evasion. I employ data from a sample of anonymized firms' tax filings
and INEGI's EAIM annual production survey, in the case of Mexico. For
Colombia, I use the EAM annual survey on manufacturing firms. In the
estimation framework, I jointly model the firm-level productivity and
the tax evasion, allowing for classical measurement error. The key
identifying assumption is that while tax evasion depends on the expected
return to cheating, productivity follows a Markov process, and
measurement error is uncorrelated with inputs and across time. In turn,
the expected returns to cheating depend on the firm's
characteristics\footnote{Tax evasion might also depend on the firm's
  market characteristics, e.g., market concentration, end-consumer
  vs.~intermediate goods market, etc.}. In particular, larger firms are
less likely to underreport inputs because of their higher cheating cost,
their higher probability to be anonymously denounced by an employee, and
their access to complex tools to legally \emph{avoid} taxes\footnote{I
  follow the literature by referring to evasion as illegal actions to
  reduce tax liability, while avoidance refers to legal actions to
  reduce tax.}.

By jointly modeling tax evasion and productivity, this paper departs
from the literature in estimating a latent variable that follows an
incentive constraint (IC) ---tax compliance---, and in focusing on the
productivity bias ---rather than on an input's coefficient. Recent work
has studied the effect of measurement error in capital inputs on
production function estimation. Their approach has been using an IV
\citep{Collard2020} or a centering condition \emph{à la} \citet{Hu2008}
(Charlotte's JMP). However, systematic misreporting is different from
measurement error in that it goes only one way and it follows an IC, as
discussed above. Likewise, because capital is not a flexible
input\footnote{In the literature, a flexible input is neither dynamic
  ---current input choice depends on its lagged value---, nor
  predetermined ---completely defined in the past.}, the focus of these
papers is on the effect of the marginal return to capital coefficient
rather than on the effect on productivity estimates.

Moreover, I focus on gross-output production functions, as opposed to
value-added, to reduce the noise that might be introduced by firms
cheating also on prices. Besides the fact that prices are an equilibrium
object and might be affected by demand also ---and not only by
supply---, firms might also artificially increase the price of inputs or
reduce the price of sales to reduce declared profits. Then, in that
sense, the tax-evasion bias magnitude of productivity estimates obtained
through a value-added production function would be greater than through
the gross-output production function. In the latter, only the price of
the flexible input ---intermediates---, and the output sale price enter
through the first-order conditions (FOC) of the profit maximization
problem of the firm\footnote{If the cross-derivative of the flexible
  input with respect to the non-flexible inputs is not zero, then
  systematic misreporting in the non-flexible inputs will also affect
  productivity.}. In contrast, the prices of the non-flexible inputs,
---capital and labor---, might introduce noise in the case of the
value-added production function.
\fxnote{tax evasion underreporting sales: use firms that sell to other firms because of opposite incentives in contrast to firm-end consumer sales. Unskilled labor might be flexible but it's the least productive, so less damage.}

\end{anfxnote}

\section{References}\label{references}

\renewcommand{\bibsection}{}
\bibliography{biblio/export.bib,biblio/export2.bib,biblio/export3.bib,biblio/export31072022.bib,biblio/b100422.bib,biblio/b270123.bib,biblio/b100424.bib}




\end{document}
